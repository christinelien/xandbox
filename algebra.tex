\documentclass{ximera}

\title{Sample}
\author{Christine Lien}

\begin{document}

\begin{abstract}
This is a test. 
\end{abstract}
\maketitle

Consider $f(x)=mx+b$. 

\[\graph{m=1, b=1, f(x)=mx+b}\]

Move the parameters $m$ and $b$ around to see how the graph changes. 
Determine all values of $m$ and $b$ that make $f(x)$ a 1-1 function. 
\begin{exercise} 
\hfill
\begin{enumerate}
\item For what value of $m$ is $mx+b$ not a function? $m=$$\answer{0}$



\item Write down the expression for $f^{-1}(x)$ in those cases. 

Write $f^{-1}(x)= \answer{\frac{1}{m}(x-b)}$

\end{enumerate}

\end{exercise}

\begin{exercise}
Consider $h(x)=\frac{1}{ax^2+b}$. 

\[\graph{a=1, b=1, h(x)=\frac{1}{ax^2+b}}\]

\begin{enumerate}
\item Find the values of $a$ and $b$ so that $h(x)$ is not a function. 

This happens when $a=\answer{0} \textrm{ and } b=\answer{0}.$

\item For all values of $a$ and $b$, $h(x)$ is not a 1-1 function. Explain why not.


\item Find the values of $a$ and $b$ so that $h(x)$ is a constant function. 

This happens whenever $a=\answer{0}$  and when $b\neq\answer{0}$

\item In most cases of $a$ and $b$, $h(x)$ is a non-constant function. It is possible in those situations to restrict the domain of $h(x)$ so that it is 1-1. Find the necessary restriction on the domain. 

\begin{itemize}
\item Domain of $h(x)$ should be $\answer{(0, \infty)}$.

\item Find the inverse. 

The inverse is $h^{-1}(x) = \answer{\sqrt{\frac{bx-1}{ax}}}$

\end{itemize}
\end{enumerate} 

\end{exercise}


\end{document}